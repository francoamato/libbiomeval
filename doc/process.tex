%
% Process API
%
\chapter{Process Information and Control}
\label{chp-process}

The \namespace{Process} package is a set of APIs used to gather information on
a process, limit the capabilities of a process, and to manage the life cycle
of processes.

\section{Process Statistics}
\label{sec-process_statistics}

When a application is running, there may be a need to obtain information of the
process executing that application. The \class{Process::Statistics} class can
be used by the application itself to gather statistics related to the current
amount of memory being used, the number of threads, and other items. In
addition, CPU user and system times can be gathered for all threads belonging
to the process.

Biometric evaluation test
drivers are linked against a third party library, and therefore, the application
writer does not control the thread count or memory usage for much of the
processing. \lstref{lst:processstatisticsuse} shows how an application can
use the \class{Statistics} API.

\begin{lstlisting}[caption={Gathering Process Statistics}, label=lst:processstatisticsuse]
#include <be_error_exception.h>
#include <be_process_statistics.h>
using namespace BiometricEvaluation;

int main(int argc, char *argv[])
{
    Process::Statistics stats{};
    cout << "success.\n";

    uint64_t userstart, userend;
    int64_t diff;
    try {
        /*
         * Obtain the user time needed to run some code ...
         */
        std::tie(userstart, std::ignore) = stats.getCPUTimes();
        cout << "Total User time at start: " << userstart << " : ";

        // Do some long processing....

        std::tie(userend, std::ignore) = stats.getCPUTimes();
        cout << "At end: " << userend << ": ";
        diff = userend - userstart;
	cout << "User time elapsed is " << diff << endl;

        /*
         * Obtain the memory usage of the current process ...
         */
        uint64_t vmrss, vmsize, vmpeak, vmdata, vmstack;
        std::tie(vmrss, vmsize, vmpeak, vmdata, vmstack) = stats.getMemorySizes();
        cout << "\tRSS: " << vmrss;
        cout << " : Size: " << vmsize;
        cout << " : Peak: " << vmpeak;
        cout << " : Data: " << vmdata;
        cout << " : Stack: " << vmstack << endl;

        /*
         * Obtain the user and system times for all threads.
         */
        auto allStats = logstats->getTasksStats();
        for (auto [tid, utime, stime]: allStats) {
                cout << "TID is " << tid <<
                " utime is " << utime <<
                ", stime is " << stime << '\n';
        }
    } catch (Error::Exception) {
	cout << "Caught " << e.getInfo() << endl;
    }

}
\end{lstlisting}

In addition to using the \class{Statistics} API to gather statistics to be
returned from
the function call, the API provides a means to have a set of
statistics logged either synchronously or asynchronously to a pair of
\class{Logsheet} objects. (See~\secref{sec-logging}) In addition, these sheets
can be contained within a \class{LogCabinet}. Applications can
start and stop logging at will to these sheets. Post mortem analysis
can then be done on the entries in the log.
\lstref{lst:processstatisticslogging} shows the use of logging.

If the \class{Statistics} object is constructed with a \class{LogCabinet}
object, the log sheets will have a file name constructed from the
process name (i.e. the application executable) and the process ID.
Alternatively, the \class{Statistics} object can be constructed with one or
two \class{Logsheet} objects, giving the application more control over where
the files are stored.

An example log sheet contains this information at the start:

\begin{verbatim}
Description: Statistics for test_be_process_statistics (PID 28370)
# Entry Usertime Systime RSS VMSize VMPeak VMData VMStack Threads
E0000000001 728889 6998 1788 57472 62612 31020 84 1
E0000000002 1300802 6998 1792 57472 62612 31020 84 1
\end{verbatim}

If the optional task statistics are to be logged, a separate file is created
in the \class{LogCabinet} with an appropriate name. Entries in this sheet are
of this form:

\begin{verbatim}
Description: Statistics for all tasks under test_be_process_statistics (PID 29193)
# Parent-ID {task-ID utime stime} ...
# Statistics auto-logger task is marked with (L)
E 0000000001 29193 {29193, 0.25, 0}
E 0000000002 29193 {29193, 0.5, 0}
E 0000000003 29193 {29193, 0.73, 0}
E 0000000004 29193 {29193, 0.96, 0}
E 0000000005 29193 {29193, 1.2, 0}
E 0000000006 29193 {29193, 1.43, 0}
# Autolog started. Interval: 1000000 microseconds.
E 0000000007 29193 {29193, 1.67, 0} {29197(L), 0, 0}
E 0000000008 29193 {29193, 1.67, 0} {29197(L), 0, 0}
E 0000000009 29193 {29193, 1.67, 0} {29197(L), 0, 0}
E 0000000010 29193 {29193, 1.67, 0} {29197(L), 0, 0}
E 0000000011 29193 {29193, 1.67, 0} {29197(L), 0, 0}
E 0000000012 29193 {29193, 1.67, 0} {29197(L), 0, 0}
E 0000000013 29193 {29193, 1.67, 0} {29197(L), 0, 0} {29198, 0, 0} {29199, 0.01, 0}
E 0000000014 29193 {29193, 1.67, 0} {29197(L), 0, 0} {29198, 0.19, 0} {29199, 0.19, 0}
E 0000000015 29193 {29193, 1.67, 0} {29197(L), 0, 0} {29198, 0.38, 0.01} {29199, 0.4, 0}
\end{verbatim}

The above example contains entries for the single task under the process, then
once auto-logging has started, more threads were created. Note that the thread
responsible for logging is identified with the {\tt (L)} tag.

When the \class{LogCabinet} is used, the \class{Statistics} object creates the
\class{Logsheet} objects with an appropriate description and comment entry with 
column headers. Each gathering of the statistics results in a single log entry.

\begin{lstlisting}[caption={Logging Process Statistics}, label=lst:processstatisticslogging]
#include <be_error_exception.h>
#include <be_io_logcabinet.h>
#include <be_process_statistics.h>
using namespace BiometricEvaluation;

int main(int argc, char *argv[])
{
    std::shared_ptr<IO::FileLogCabinet> lc;
    lc.reset(new IO::FileLogCabinet("statLogCabinet", "Cabinet for Stats"));
    std::unique_ptr<Process::Statistics> logstats;

    try {
	// Auto-log both process and task statistics
	logstats.reset(new Process::Statistics(lc, true));
    } catch (Error::Exception &e) {
	cout << "Caught " << e.getInfo() << endl;
	return (EXIT_FAILURE);
    }
    try {
	while (some_processing_to_do) {
	    // Do the work
	    // Synchronously log after the work is done.
	    logstats->logStats();
	}
    } catch (Error::Exception &e) {
	cout << "Caught " << e.getInfo() << endl;
	return (EXIT_FAILURE);
    }

    // Set up asynchronous logging, every 1/10 second
    try {
	logstats->startAutoLogging(100000);
    } catch (Error::ObjectExists &e) {
	cout << "Caught " << e.getInfo() << endl;
	return (EXIT_FAILURE);
    }

    // Do some other work

    // Stop logging
    logstats->stopAutoLogging();
}
\end{lstlisting}

\section{Process Management}
\label{sec-process_management}

During a biometric evaluation or other long-running CPU-bound task, it's 
beneficial to make efficient use of all the hardware available on the system.
Applications can take advantage of a multi-core machine, for example.
\sname\ aims to simply this by abstracting the usage of process and thread
creation to run multiple instances of the same function in parallel.

\subsection{Manager}
\label{subsec-process_manager}

There are three class hierarchies involved in the abstraction.  
The \class{BiometricEvaluation::Process::Manager} classes control the
technique of process manipulation that will be used.  \sname\ provides two
example abstractions: \class{Fork\allowbreak Manager} and
\class{POSIX\allowbreak Thread\allowbreak Manager}.  When
using \class{Fork\allowbreak Manager}, new processes will be created with 
\code{fork(2)}, with mediated access to these new processes through the
\class{Manager}.  Likewise, \class{POSIX\allowbreak Thread\allowbreak Manager}
creates new POSIX threads.  Because both of theses classes inherit from
\class{Manager}, it is as trivial as changing the \class{Manager} object type
to change how the workload is made parallel.

\subsection{Worker}
\label{subsec-process_worker}

In the application using a \class{Manager}, a \class{Worker} subclass must be
implemented.  An example \class{Worker} is shown in
\lstref{lst:process_worker-example}.
The entry-point for a \class{Worker} is the \code{worker\allowbreak Main()}
method, which must be implemented by the client application.  Although
\code{worker\allowbreak Main()} takes no arguments, data may be transmitted
into the object through
\class{Worker\allowbreak Controller}'s~(\ref{subsec-process_workercontroller})
\code{set\allowbreak Parameter()} method.  Within the \class{Worker} instance,
the parameters are then retrieved with \code{get\allowbreak Parameter()} when
provided with the unique parameter name.

A responsible worker performs its operations as fast as it can.
However, at any given time, the manager may ask the worker to
stop.  It then becomes the {\em responsibility of the worker} to 
stop as soon as possible.  The \class{Worker} is notified of the stop request
through its \code{stop\allowbreak Requested()} method.  Note that the
manager does \textbf{not} force the worker to stop, though
prolonged work or cleanup in the worker would likely produce
undesired results in the client application.  As such, a responsible
worker checkpoints itself to prepare for premature stops requested by
the manager. While it is important for a worker to stop as soon
as possible after the request is received, it is also important not to leave
work in an unsynchronized state.  In \lstref{lst:process_worker-example},
notice how
the \class{Employee} must continue the interaction with the \class{Customer}
before a stop request is handled, even if the \class{Employee}'s shift has
ended.  Leaving the method before the \class{Customer}'s order has been
delivered would leave the \class{Customer} object in an unsafe state (hungry). 

\begin{lstlisting}[caption={A Responsible \class{Worker} Implementation}, label=lst:process_worker-example]
#include <cstdlib>
#include <tr1/memory>
#include <queue>

#include <restaurant.h>

#include <be_process_forkmanager.h>

using namespace std;
using namespace BiometricEvaluation;
using namespace Restaurant;

class ResponsibleEmployeeTask : public Process::Worker
{
public:
	int32_t
	workerMain()
	{
		int32_t status = EXIT_FAILURE;
		
		/* Retrieve objects assigned to this Task */
		tr1::shared_ptr<Employee> employee =
		    tr1::static_pointer_cast<Employee>(
		    this->getParameter("employee"));
		tr1::shared_ptr< queue<Customer*> > customers = 
		    tr1::static_pointer_cast< queue<Customer*> >(
		    this->getParameter("customers")
		
		employee->clockIn();
		
		Customer *customer;
		/* Checkpoint after each customer */
		while (this->stopRequested() == false ||
		    employee->isShiftOver() == false) {
			customer = customers->front();
			
			if (customer != NULL) {
				employee->takeOrder(customer);
				employee->cookFood(customer);
				employee->deliverOrder(customer);
				
				customers->pop();
			}
		}
		
		employee->settleCashDrawer();
		employee->clockOut();
		
		status = EXIT_SUCCESS;
		return (status);
	}
	~ResponsibleEmployeeTask() {}
};
\end{lstlisting}

After a manager starts its workers, the manager has the
option of waiting until all \class{Worker}s exit \code{worker\allowbreak Main()}
before continuing code execution.  If not waiting,
there are several methods the manager can perform to keep track of the status
of the workers.  Even if not waiting for workers to return,
a responsible manager will wait a reasonable amount of time for
workers to \code{return} before application termination.  An example of this
reasonable waiting period can be seen in \lstref{lst:process_manager-example}.

\subsection{WorkerController}
\label{subsec-process_workercontroller}

The final piece of the process management puzzle is the
\class{Worker\allowbreak Controller} hierarchy.  This class decorates and
mediates communication between the \class{Manager} and the \class{Worker}.  
\class{Worker\allowbreak Controller} objects may only be instantiated by
a \class{Manager} object.  All communications to the \class{Worker}
(e.g. \code{is\allowbreak Working()}) should be delegated through the
\class{Worker\allowbreak Controller}.  If defining a new \class{Manager}, note
that the \class{Worker\allowbreak Controller} may seem unnecessary for the
parallel technique being employed.  It's true
that some parallel techniques may not require this ``middle-man''
approach, but others do.  Do not be concerned if a \class{Worker\allowbreak
Controller} implementation ends up being nothing more than a ``pass-thru'' to
the \class{Worker}.

\lstref{lst:process_manager-example} is a continuation of
\lstref{lst:process_worker-example} demonstrating the use of \class{Manager}s
and \class{Worker\allowbreak Controller}s.

\begin{lstlisting}[caption={Using \class{Manager}s and \class{WorkerController}s}, label=lst:process_manager-example]
int
main(
    int argc,
    char *argv[])
{
	static const uint32_t numEmployees = 3;
	int status = EXIT_FAILURE;
	
	tr1::shared_ptr<Process::Manager> shiftLeader(new Process::ForkManager);
	queue<Customer*> *customers = new queue<Customer*>();
	
	/* Create Employees (Workers/WorkerControllers) */
	tr1::shared_ptr<Process::WorkerController> employees[numEmployees];
	for (uint32_t i = 0; i < numEmployees; i++) {
		employees[i] = shiftLeader->addWorker(
		    tr1::shared_ptr<ResponsibleEmployeeTask>(
		    new ResponsibleEmployeeTask()));
		
		/* Assign employees to each Task */
		employees[i]->setParameter("employee",
		    tr1::shared_ptr<Employee>(new Employee()));
		employees[i]->setParameter("customers",
		    tr1::shared_ptr< queue<Customer*> >(customers);
	}
	
	/* Employees start serving customers while shift leader manages */
	shiftLeader->startWorkers(false);
	
	/* Customers enter the queue... */
	queue<Restaurant::AdministrativeTasks> adminTasks;
	adminTasks.push("Inventory");
	adminTasks.push("Customer Complaints");
	adminTasks.push("Clean Dining Room");
	
	while (shiftLeader->getNumActiveWorkers() != 0) {
		shiftLeader->doTask(adminTasks.front());
		adminTasks.pop();
	}
	
	/* ...end of the day */
	for (uint32_t i = 0; i < numEmployees; i++)
		if (employees[i]->isWorking())
			shiftLeader->stopWorker(employees[i]);
	
	/* 
	 * Wait a reasonable amount of time before locking up for the night
	 * (in this case, indefinitely).
	 */
	while (shiftLeader->getNumActiveWorkers() > 0)
		sleep(1);
		
	shiftLeader->armAlarmAndExit();
	
	status = EXIT_SUCCESS;	
	return (status);
}

\end{lstlisting}

\subsection{Communications}
\label{subsec:communications}

Managers and workers may have a good reason to send and receive
messages directly.  A communications mechanism is built-in to the
\nameref{sec-process_management} model to facilitate such communications. 
The type and content of the message is completely up to the client
implementation, since messages are sent as \class{AutoArray}s.
A manager does not directly send messages to a
worker.  This service is provided by the \class{Worker\-Controller}
(via \texttt{send\-Message\-To\-Worker()}).

Managers can keep an eye on incoming messages by calling the
(optionally blocking) \texttt{wait\-For\-Message()} method.  This method will
return a handle to the worker that sent a message.  Alternatively, the
manager can invoke \texttt{get\-Next\-Message()} (again, blocking 
optional) to immediately receive the next message.

\lstref{lst:process-worker-communication-example} and 
\lstref{lst:process-manager-communication-example} are continuations of
\lstref{lst:process_worker-example} and \lstref{lst:process_manager-example}
respectively, showing an example of communication, using \class{std::string}
messages.

\begin{lstlisting}[caption={\class{Worker} Communication}, label=lst:process-worker-communication-example]
	Memory::uint8Array msg;

	/* Deal with next customer unless Manager interrupts in next second */
	if (this->waitForMessage(1)) {
		if (this->receiveMessageFromManager(msg)) {
			Action action = Restaurant::messageToAction(msg);
			switch (action) {
			case TAKE_BREAK:
				employee->goOnBreak();
				break;
			/* ... */
			}
		}
	}
	
	/* ... */
	
	if (customer->isComplaining()) {
		sprintf((char *)&(*msg), "Customer Complant");
		this->sendMessageToManager(msg);
	}
\end{lstlisting}
\begin{lstlisting}[caption={\class{Manager} Communication}, label=lst:process-manager-communication-example]
	tr1::shared_ptr<Process::WorkerController> sender;
	Memory::uint8Array msg;
	
	/* Do routine tasks unless employee has concern in the next 2 seconds */
	while (this->getNextMessage(sender, msg, 2)) {
		Action action = Restaurant::messageToAction(msg);
		switch (action) {
		case CUSTOMER_COMPLAINT:
			sprintf((char *)&(*msg), "I'll take care of it.");
			this->sendMessageToWorker(msg);
			break;
		/* ... */
		}
	}
	
	/* ... */
	
	/*  Closing Time */
	sprintf((char *)&(*msg), "Clock out and go home.");
	this->broadcastMessage(msg);
\end{lstlisting}
